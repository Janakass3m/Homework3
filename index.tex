% Options for packages loaded elsewhere
\PassOptionsToPackage{unicode}{hyperref}
\PassOptionsToPackage{hyphens}{url}
%
\documentclass[
]{article}
\usepackage{amsmath,amssymb}
\usepackage{iftex}
\ifPDFTeX
  \usepackage[T1]{fontenc}
  \usepackage[utf8]{inputenc}
  \usepackage{textcomp} % provide euro and other symbols
\else % if luatex or xetex
  \usepackage{unicode-math} % this also loads fontspec
  \defaultfontfeatures{Scale=MatchLowercase}
  \defaultfontfeatures[\rmfamily]{Ligatures=TeX,Scale=1}
\fi
\usepackage{lmodern}
\ifPDFTeX\else
  % xetex/luatex font selection
\fi
% Use upquote if available, for straight quotes in verbatim environments
\IfFileExists{upquote.sty}{\usepackage{upquote}}{}
\IfFileExists{microtype.sty}{% use microtype if available
  \usepackage[]{microtype}
  \UseMicrotypeSet[protrusion]{basicmath} % disable protrusion for tt fonts
}{}
\makeatletter
\@ifundefined{KOMAClassName}{% if non-KOMA class
  \IfFileExists{parskip.sty}{%
    \usepackage{parskip}
  }{% else
    \setlength{\parindent}{0pt}
    \setlength{\parskip}{6pt plus 2pt minus 1pt}}
}{% if KOMA class
  \KOMAoptions{parskip=half}}
\makeatother
\usepackage{xcolor}
\usepackage[margin=1in]{geometry}
\usepackage{color}
\usepackage{fancyvrb}
\newcommand{\VerbBar}{|}
\newcommand{\VERB}{\Verb[commandchars=\\\{\}]}
\DefineVerbatimEnvironment{Highlighting}{Verbatim}{commandchars=\\\{\}}
% Add ',fontsize=\small' for more characters per line
\usepackage{framed}
\definecolor{shadecolor}{RGB}{248,248,248}
\newenvironment{Shaded}{\begin{snugshade}}{\end{snugshade}}
\newcommand{\AlertTok}[1]{\textcolor[rgb]{0.94,0.16,0.16}{#1}}
\newcommand{\AnnotationTok}[1]{\textcolor[rgb]{0.56,0.35,0.01}{\textbf{\textit{#1}}}}
\newcommand{\AttributeTok}[1]{\textcolor[rgb]{0.13,0.29,0.53}{#1}}
\newcommand{\BaseNTok}[1]{\textcolor[rgb]{0.00,0.00,0.81}{#1}}
\newcommand{\BuiltInTok}[1]{#1}
\newcommand{\CharTok}[1]{\textcolor[rgb]{0.31,0.60,0.02}{#1}}
\newcommand{\CommentTok}[1]{\textcolor[rgb]{0.56,0.35,0.01}{\textit{#1}}}
\newcommand{\CommentVarTok}[1]{\textcolor[rgb]{0.56,0.35,0.01}{\textbf{\textit{#1}}}}
\newcommand{\ConstantTok}[1]{\textcolor[rgb]{0.56,0.35,0.01}{#1}}
\newcommand{\ControlFlowTok}[1]{\textcolor[rgb]{0.13,0.29,0.53}{\textbf{#1}}}
\newcommand{\DataTypeTok}[1]{\textcolor[rgb]{0.13,0.29,0.53}{#1}}
\newcommand{\DecValTok}[1]{\textcolor[rgb]{0.00,0.00,0.81}{#1}}
\newcommand{\DocumentationTok}[1]{\textcolor[rgb]{0.56,0.35,0.01}{\textbf{\textit{#1}}}}
\newcommand{\ErrorTok}[1]{\textcolor[rgb]{0.64,0.00,0.00}{\textbf{#1}}}
\newcommand{\ExtensionTok}[1]{#1}
\newcommand{\FloatTok}[1]{\textcolor[rgb]{0.00,0.00,0.81}{#1}}
\newcommand{\FunctionTok}[1]{\textcolor[rgb]{0.13,0.29,0.53}{\textbf{#1}}}
\newcommand{\ImportTok}[1]{#1}
\newcommand{\InformationTok}[1]{\textcolor[rgb]{0.56,0.35,0.01}{\textbf{\textit{#1}}}}
\newcommand{\KeywordTok}[1]{\textcolor[rgb]{0.13,0.29,0.53}{\textbf{#1}}}
\newcommand{\NormalTok}[1]{#1}
\newcommand{\OperatorTok}[1]{\textcolor[rgb]{0.81,0.36,0.00}{\textbf{#1}}}
\newcommand{\OtherTok}[1]{\textcolor[rgb]{0.56,0.35,0.01}{#1}}
\newcommand{\PreprocessorTok}[1]{\textcolor[rgb]{0.56,0.35,0.01}{\textit{#1}}}
\newcommand{\RegionMarkerTok}[1]{#1}
\newcommand{\SpecialCharTok}[1]{\textcolor[rgb]{0.81,0.36,0.00}{\textbf{#1}}}
\newcommand{\SpecialStringTok}[1]{\textcolor[rgb]{0.31,0.60,0.02}{#1}}
\newcommand{\StringTok}[1]{\textcolor[rgb]{0.31,0.60,0.02}{#1}}
\newcommand{\VariableTok}[1]{\textcolor[rgb]{0.00,0.00,0.00}{#1}}
\newcommand{\VerbatimStringTok}[1]{\textcolor[rgb]{0.31,0.60,0.02}{#1}}
\newcommand{\WarningTok}[1]{\textcolor[rgb]{0.56,0.35,0.01}{\textbf{\textit{#1}}}}
\usepackage{graphicx}
\makeatletter
\newsavebox\pandoc@box
\newcommand*\pandocbounded[1]{% scales image to fit in text height/width
  \sbox\pandoc@box{#1}%
  \Gscale@div\@tempa{\textheight}{\dimexpr\ht\pandoc@box+\dp\pandoc@box\relax}%
  \Gscale@div\@tempb{\linewidth}{\wd\pandoc@box}%
  \ifdim\@tempb\p@<\@tempa\p@\let\@tempa\@tempb\fi% select the smaller of both
  \ifdim\@tempa\p@<\p@\scalebox{\@tempa}{\usebox\pandoc@box}%
  \else\usebox{\pandoc@box}%
  \fi%
}
% Set default figure placement to htbp
\def\fps@figure{htbp}
\makeatother
\setlength{\emergencystretch}{3em} % prevent overfull lines
\providecommand{\tightlist}{%
  \setlength{\itemsep}{0pt}\setlength{\parskip}{0pt}}
\setcounter{secnumdepth}{-\maxdimen} % remove section numbering
\usepackage{bookmark}
\IfFileExists{xurl.sty}{\usepackage{xurl}}{} % add URL line breaks if available
\urlstyle{same}
\hypersetup{
  pdftitle={Homework3},
  hidelinks,
  pdfcreator={LaTeX via pandoc}}

\title{Homework3}
\author{}
\date{\vspace{-2.5em}2025-09-25}

\begin{document}
\maketitle

\section{Should We Change the
Design?}\label{should-we-change-the-design}

I would recommend that this company commit to redesigning their website.

\begin{Shaded}
\begin{Highlighting}[]
\CommentTok{\# load in data}
\NormalTok{data }\OtherTok{\textless{}{-}} \FunctionTok{read.csv}\NormalTok{(}\StringTok{"homework3\_data.csv"}\NormalTok{)}
\end{Highlighting}
\end{Shaded}

\section{Supporting Evidence}\label{supporting-evidence}

\subsubsection{Getting the color palette from
Target}\label{getting-the-color-palette-from-target}

\begin{Shaded}
\begin{Highlighting}[]
\FunctionTok{library}\NormalTok{(colorfindr)}
\CommentTok{\# Target color palette}
\NormalTok{dat }\OtherTok{\textless{}{-}} \FunctionTok{get\_colors}\NormalTok{(}\StringTok{"/Users/janak/Homework3/OtherWebsite.png"}\NormalTok{)}
\NormalTok{cols }\OtherTok{\textless{}{-}} \FunctionTok{make\_palette}\NormalTok{(dat[}\DecValTok{1}\SpecialCharTok{:}\DecValTok{100}\NormalTok{, ])}
\end{Highlighting}
\end{Shaded}

\pandocbounded{\includegraphics[keepaspectratio]{index_files/figure-latex/unnamed-chunk-1-1.pdf}}

\begin{Shaded}
\begin{Highlighting}[]
\NormalTok{cols}
\end{Highlighting}
\end{Shaded}

\begin{verbatim}
##  [1] "#FFFFFF" "#BB271A" "#F1EFE7" "#333333" "#BEE3FA" "#BAE1F8" "#CDE8FB"
##  [8] "#B3DDF9" "#E9F5FE" "#A5D6F7"
\end{verbatim}

\subsection{Part A (Supporting
Evidence)}\label{part-a-supporting-evidence}

\begin{Shaded}
\begin{Highlighting}[]
\FunctionTok{library}\NormalTok{(dplyr)}
\FunctionTok{library}\NormalTok{(ggplot2)}

\NormalTok{df }\OtherTok{\textless{}{-}}\NormalTok{ data }\SpecialCharTok{\%\textgreater{}\%}
  \FunctionTok{mutate}\NormalTok{(}\AttributeTok{design\_f =} \FunctionTok{factor}\NormalTok{(design, }\AttributeTok{labels =} \FunctionTok{c}\NormalTok{(}\StringTok{"Old"}\NormalTok{, }\StringTok{"New"}\NormalTok{)))}

\CommentTok{\# Target palette}
\NormalTok{pal\_cat }\OtherTok{\textless{}{-}} \FunctionTok{c}\NormalTok{(}\StringTok{"Old"} \OtherTok{=} \StringTok{"\#B3DDF9"}\NormalTok{, }\StringTok{"New"} \OtherTok{=} \StringTok{"\#BB271A"}\NormalTok{)}

\CommentTok{\# Boxplot to show mean differences between old and new}
\FunctionTok{ggplot}\NormalTok{(df, }\FunctionTok{aes}\NormalTok{(}\AttributeTok{x =}\NormalTok{ design\_f, }\AttributeTok{y =}\NormalTok{ sales, }\AttributeTok{fill =}\NormalTok{ design\_f)) }\SpecialCharTok{+}
  \FunctionTok{geom\_boxplot}\NormalTok{() }\SpecialCharTok{+}
  \FunctionTok{scale\_fill\_manual}\NormalTok{(}\AttributeTok{values =}\NormalTok{ pal\_cat, }\AttributeTok{guide =} \StringTok{"none"}\NormalTok{) }\SpecialCharTok{+}
  \FunctionTok{labs}\NormalTok{(}\AttributeTok{title =} \StringTok{"Sales by Design"}\NormalTok{,}
       \AttributeTok{x =} \StringTok{"Design"}\NormalTok{, }\AttributeTok{y =} \StringTok{"Sale Amount ($)"}\NormalTok{) }\SpecialCharTok{+}
  \FunctionTok{theme\_grey}\NormalTok{()}
\end{Highlighting}
\end{Shaded}

\pandocbounded{\includegraphics[keepaspectratio]{index_files/figure-latex/unnamed-chunk-2-1.pdf}}

As we can see the new design has a higher median sales amount than the
old design, supporting that they should change their design.

\begin{Shaded}
\begin{Highlighting}[]
\CommentTok{\# Distribution plot to show how the new design has shifted right}
\FunctionTok{ggplot}\NormalTok{(df, }\FunctionTok{aes}\NormalTok{(}\AttributeTok{x =}\NormalTok{ sales, }\AttributeTok{fill =}\NormalTok{ design\_f)) }\SpecialCharTok{+}
  \FunctionTok{geom\_histogram}\NormalTok{(}\AttributeTok{alpha =}\NormalTok{ .}\DecValTok{4}\NormalTok{, }\AttributeTok{position =} \StringTok{"identity"}\NormalTok{, }\AttributeTok{bins =} \DecValTok{20}\NormalTok{) }\SpecialCharTok{+}
  \FunctionTok{scale\_fill\_manual}\NormalTok{(}\AttributeTok{values =}\NormalTok{ pal\_cat, }\AttributeTok{name =} \StringTok{"Design"}\NormalTok{) }\SpecialCharTok{+}
  \FunctionTok{labs}\NormalTok{(}\AttributeTok{title =} \StringTok{"Distribution of Sales by Design"}\NormalTok{,}
       \AttributeTok{x =} \StringTok{"Sale Amount ($)"}\NormalTok{, }\AttributeTok{y =} \StringTok{"Count"}\NormalTok{)}
\end{Highlighting}
\end{Shaded}

\pandocbounded{\includegraphics[keepaspectratio]{index_files/figure-latex/pressure-1.pdf}}

This graph supports the premise because we can see that the new design
distribution is shifted to the right meaning higher sales.

\subsection{Part B (Estimate the
difference)}\label{part-b-estimate-the-difference}

\begin{Shaded}
\begin{Highlighting}[]
\CommentTok{\# calc the mean diff between averages}

\NormalTok{mean\_old }\OtherTok{=} \FunctionTok{mean}\NormalTok{(df}\SpecialCharTok{$}\NormalTok{sales[df}\SpecialCharTok{$}\NormalTok{design\_f }\SpecialCharTok{==} \StringTok{"Old"}\NormalTok{])}
\NormalTok{mean\_new }\OtherTok{=} \FunctionTok{mean}\NormalTok{(df}\SpecialCharTok{$}\NormalTok{sales[df}\SpecialCharTok{$}\NormalTok{design\_f }\SpecialCharTok{==} \StringTok{"New"}\NormalTok{])}
\NormalTok{diff }\OtherTok{=}\NormalTok{ mean\_new }\SpecialCharTok{{-}}\NormalTok{ mean\_old}

\FunctionTok{c}\NormalTok{(}\AttributeTok{mean\_old =}\NormalTok{ mean\_old, }\AttributeTok{mean\_new =}\NormalTok{ mean\_new, }\AttributeTok{diff =}\NormalTok{ diff)}
\end{Highlighting}
\end{Shaded}

\begin{verbatim}
##  mean_old  mean_new      diff 
## 31.848190 35.513095  3.664904
\end{verbatim}

The average difference between the old and new design is \$3.66 in favor
of the new design.

\subsection{Part C (check \$1.80)}\label{part-c-check-1.80}

\begin{Shaded}
\begin{Highlighting}[]
\CommentTok{\# Do a t test to show the difference is significant}
\NormalTok{t\_test }\OtherTok{\textless{}{-}} \FunctionTok{t.test}\NormalTok{(df}\SpecialCharTok{$}\NormalTok{sales[df}\SpecialCharTok{$}\NormalTok{design}\SpecialCharTok{==}\DecValTok{1}\NormalTok{],}
\NormalTok{                    df}\SpecialCharTok{$}\NormalTok{sales[df}\SpecialCharTok{$}\NormalTok{design}\SpecialCharTok{==}\DecValTok{0}\NormalTok{],}
                    \AttributeTok{alternative =} \StringTok{"greater"}\NormalTok{,}
                    \AttributeTok{mu =} \FloatTok{1.8}\NormalTok{)}

\NormalTok{t\_test}
\end{Highlighting}
\end{Shaded}

\begin{verbatim}
## 
##  Welch Two Sample t-test
## 
## data:  df$sales[df$design == 1] and df$sales[df$design == 0]
## t = 4.1499, df = 186.01, p-value = 2.528e-05
## alternative hypothesis: true difference in means is greater than 1.8
## 95 percent confidence interval:
##  2.922037      Inf
## sample estimates:
## mean of x mean of y 
##  35.51309  31.84819
\end{verbatim}

Null hypothesis (H₀): The mean increase in sales from the new design
compared to the old design is less than or equal to \$1.80. Alternative
hypothesis (Hₐ): The mean increase in sales from the new design compared
to the old design is greater than \$1.80.

Since the p-value (2.53e-05) is less than 0.05, we reject the null
hypothesis, providing strong evidence that switching to the new design
leads to a significant increase in sales beyond the \$1.80 threshold.

\section{Alternative Statement}\label{alternative-statement}

The redesign should not be implemented because it does not lead to a
meaningful increase in sales beyond \$1.80 per customer.

\end{document}
